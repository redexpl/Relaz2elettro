\documentclass{article}
\usepackage{graphicx}
\usepackage[utf8]{inputenc}
\author{Alessandro Rosso, Elia Migliore, Franco Ruggeri, Marco Riggio}
\title{Relazione Seconda Esercitazione Sistemi Elettronici}
\date{15 December 2017}

\begin{document}

\section{Introduzione}
In questa esercitazione sperimentale di sistemi elettronici abbiamo misurato il comportamento di due amplificatori reali, uno non invertente e uno invertente.\\Abbiamo quindi poi confrontare i risultati ottenuti dalle misure con quelli ottenuti dai calcoli teorici, osservando i limiti che le semplificazioni teoriche inducono.

\section{Strumenti}
Per effettuare questa esercitazione abbiamo dovuto utilizzare quindi:
\begin{itemize}
	\item \textbf{Generatore da banco}
	\item \textbf{Generatore di segnali}
    \item \textbf{Oscilloscopio digitale}
	\item \textbf{Scheda con amplificatori "A2"}
\end{itemize}

\subsection{Generatore banco}
Il generatore da banco è stato configurato in modo che erogasse 12V sia sull'uscita 1 che sull'uscita 2.\\
Si è quindi collegato un cavo con entrambi i connettori a banana in modo che cortocircuitasse il GND della prima uscita e il +12V della seconda uscita, così da poterlo utilizzare come \textbf{generatore duale}.\\ Su questo cavo banana-banana si è poi anche inserito il connettore a banana verde ( potenziale a 0V ) che usciva dal connettore collegato alla porta J8 presente sulla scheda con gli amplificatori.\\
I connettori a banana rosso e nero che uscivano dal connettore connesso a J8 sono poi stati collegati alle uscite 1 - +12V e 2 - GND del generatore da banco, rispettivamente.\\
Abbiamo così ottenuto una differenza di potenziale pari a +24V tra i cavi rosso e nero del connettore di alimentazione della scheda "A2", +12V tra il rosso e il verde e -12V tra il verde e il nero.

\subsection{Generatore di segnali}
Il generatore di segnali è stato collegato attraverso un connettore coassiale sia all'uscita 50ohm (del generatore di segnali) sia all'ingresso J1 della scheda "A2".

\subsection{Oscilloscopio digitale}
\subsubsection{Canale 1} Il canale 1 dell'oscilloscopio è stato collegato attraverso un connettore da coassiale (ingresso verso oscillosopio) a morsetti rosso e nero, collegati a J4 e J5 (della scheda "A2") rispettivamente.
\subsubsection{Canale 2} Il canale 2 dell'oscilloscopio è stato collegato sempre attraverso un connettore da coassiale (ingresso verso oscillosopio) a morsetti rosso e nero, collegati a J6 e J7 rispettivamente.

\section{Amplificatore non invertente senza elementi reattivi}
Per la prima parte dell'esercitazione siamo andati ad analizzare l'amplificatore non invertente presente sulla scheda.
\subsection{Calcoli teorici}
...
\subsection{Predisposizione scheda}
Per poter utilizzare l'amplificatore non invertente bisognava predisporre la scheda andando a commutare degli \textbf{interruttori} presenti su di essa, nello specifico abbiamo settato gli interruttori nel seguente modo:
\begin{itemize}
	\item \textbf{S1}: Posizione 2\\ \textit{Mettendo S1 in \textbf{posizione 2} si collega l'ingresso (l'uscita del generatore di segnale) all'ingresso dell'\textbf{amplificatore non invertente}.\\Viceversa se posto in \textbf{posizione 1} si collegherebbe J1 all'ingresso dell'\textbf{amplificatore invertente}.}
	\item \textbf{S2}: Posizione 2\\ \textit{Mettendo S2 in \textbf{posizione 2} si collega l'uscita dell'\textbf{amplificatore non invertente} al "circuito di uscita", viceversa se si mette S2 in \textbf{posizione 1} si collega l'uscita dell'\textbf{amplificatore invrtente} al circuito di uscita}
	\item \textbf{S3}: Posizione 2\\ \textit{Commutando S3 in \textbf{posizione 1} insieme a S4 in posizione 1 si va ad inserire il condensatore \textbf{C10} in "serie alla resistenza di ingresso" (e in parallelo a C5) dell'amplificatore}
	\item \textbf{S4}: Posizione 2\\ \textit{Commutando S4 in \textbf{posizione 1} si andrebbe a inserire \textbf{C5} in "serie alla resistenza di ingresso" (e in parallelo a C10 se S3 = 1)}
	\item \textbf{S5}: Posizione 2\\\textit{Mettendo S5 in \textbf{posizione 1} si va ad inserire la resistenza \textbf{R9} in serie al circuito di ingresso}
	\item \textbf{S6}: Posizione 1\\ \textit{Mettendo S6 in \textbf{posizione 2} sia ndrebbe ad inserire \textbf{R10} in serie al circuito di uscita}
	\item \textbf{S7}: Posizione 1\\ \textit{Mettendo S7 in \textbf{posizione 2} sia ndrebbe ad inserire \textbf{R11} in serie al circuito di uscita}
	\item \textbf{S8}: Posizione 1\\ \textit{Mettendo S8 in \textbf{posizione 2} sia ndrebbe ad inserire \textbf{C6} in serie al circuito di uscita}
	\item \textbf{S9}: Posizione 1\\ \textit{Mettendo S9 in \textbf{posizione 2} sia ndrebbe ad inserire \textbf{C9} in serie al circuito di uscita}
\end{itemize}

\subsection{Misura guadagno}
\subsubsection{Impostazione segnale generatore segnale}
Per effettuare la misura di guadagno è richiesto di impostare il generatore di segnale in modo che generi un \textbf{segnale} di uscita \textbf{sinusoidale} di \textbf{frequenza} pari a \textbf{800Hz} e \textbf{valore di picco-picco} di \textbf{1V}.

\subsubsection{Segnale misurato}
Passiamo ora all'analisi dei valori misurati attraverso l'oscilloscopio.\\Dall'analisi dell'uscita visualizzata sul display dell'oscilloscopio possiamo calcolare una \textbf{tensione di ingresso} pari a \large $V_s = (980 \pm 39)$mV \normalsize (in quanto possiamo leggere il risultato come 4.9 divisioni con una scala di 200mV a divisione).\\
Per quanto riguarda la \textbf{tensione di uscita} invece leggiamo un valore di \large $V_u = (8.40 \pm 0.36)$V \normalsize (ottenuto da 4.2 divisione con una scala di 2.00V per divisione).
\subsubsection{Calcolo guadagno}
Ottemiamo quindi un \textbf{guadagno} pari a:\\ \large $A_v = \frac{V_s}{V_u} = (8.57 \pm 0.71)$ \normalsize \\Che, espresso \textbf{in decibel} diventa:\\ \large $ |A_v|_{dB} = (18.7 \pm 1.7)$dB \normalsize

\subsection{Misura della resistenza di ingresso}
\subsubsection{Metodo di misura per le resistenze}
Per poter effettuare le misure di resistenza si è scelto di sfruttare l'inserimento di una \textbf{resistenza in serie} al circuito di ingresso, R9 ("inseribile" tramite l'interruttore S5 ???).\\Calcolando quindi la tensione di uscita senza resistenza R9 in serie si ottiene:\\
\large $V_{u} = A_{v} * V_{in}$ \normalsize \\ Invece se si inserisce R9 in serie commutando la posizione di S5???? si ottiene:\\
\large $ V_{u}^{'} = A_{v} * V_{in} * \frac{R_{in}}{R_{in} + R_9} $ \normalsize \\ Facendo il rapporto \large $\frac{V_{u}^{'}}{{V_u}}$ \normalsize si ottiene quindi: \\ \large $R_{in} = \frac{R_9 * V_{u}^{'}}{V_u - V_{u}^{'}} $ \normalsize
\subsubsection{Calcolo resistenza ingresso}
I valori numerici misurati sono i seguenti:
\begin{itemize}
	\item \large $R_{9} = 10k\Omega \pm 1\%$
	\item \large $V_{u} = (8.40 \pm 0.36)V$
	\item \large $V_{u}^{'} = (4.40 \pm 0.18)V$
\end{itemize}
Inserendoli nella formula precedente otteniamo quindi un valore di: \\ \large $R_{in} = (11.0 \pm 2.0)k\Omega$

\subsection{Misura della resistenza di uscita}
\subsubsection{Metodo di misura per le resistenze}
Per poter effettuare le misure di resistenza si è scelto di sfruttare l'inserimento di una \textbf{resistenza in serie} al circuito di uscita, R10 ("inseribile" tramite l'interruttore S6 ???).\\Calcolando quindi la tensione di uscita senza resistenza R10 in serie si ottiene:\\
\large $V_{u} = A_{v} * V_{in}$ \normalsize \\ Invece se si inserisce R10 in serie commutando la posizione di S6???? si ottiene:\\
\large $ V_{u}^{'} = A_{v} * V_{in} * \frac{R_{10}}{R_{10} + R_u} $ \normalsize \\ Facendo il rapporto \large $\frac{V_{u}^{'}}{{V_u}}$ \normalsize si ottiene quindi: \\ \large $R_{in} = R_{10} * (\frac{V_{u}-V_{u}^{'}}{V_{u}^{'}}) $ \normalsize
\subsubsection{Calcolo resistenza uscita}
I valori numerici misurati sono i seguenti:
\begin{itemize}
	\item \large $R_{10} = 1k\Omega \pm 5\%$
	\item \large $V_{u} = (8.40 \pm 0.36)V$
	\item \large $V_{u}^{'} = (4.20 \pm 0.18)V$
\end{itemize}
Inserendoli nella formula precedente otteniamo quindi un valore di: \\ \large $R_{u} = (1.0 \pm 0.23)k\Omega$

\subsection{Confronto valori teorici e valori ottenuti dalle misure}
...

\section{Amplificatore non invertente con elementi reattivi}
\subsection{Calcoli teorici}
...
\subsection{Predisposizione scheda}
Per effettuare queste nuove misure è necessario ricofigurare la posizione degli interruttori sulla scheda secondo la seguente configurazione:
\begin{itemize}
	\item S1: Posizione 2
	\item S2: Posizione 2
	\item S3: Posizione 2
	\item S4: Posizione 2
	\item S5: Posizione 2
	\item S6: Posizione 2
	\item S7: Posizione 2
	\item S8: Posizione 2
	\item S9: Posizione 2
\end{itemize}

\subsection{Misure in frequenza}
\subsubsection{Tabella delle misure}

% tabella aggiunta da elia
Tabella:\\
\begin{table}[]
\centering
\label{my-label}
\renewcommand{\arraystretch}{1.5}
\begin{tabular}{|c|c|c|c|}
\hline \large{$f$} & \large{$\Delta_\phi$} & \large{${|A_v|}_{calcolato}$} & \large{${|A_v|}_{misurato}$}
\\
\hline $300Hz$ & $(1.21 \pm 0.06)*10^{-3} rad$ & $(7.1 \pm 3.2) $ & $ (7.00 \pm 0.70) $ \\
\hline $1kHz$ & $(7.54 \pm 0.46)*10^{-4} rad$ & $(15.52 \pm 3.9)$ & $ (15.30 \pm 0.67) $ \\
\hline $3kHz$ & $(1.70 \pm 0.22)*10^{-4} rad$ & $(18.6 \pm 5.0)$ & $ (18.02 \pm 0.62) $ \\
\hline $10kHz$ & $(5.03 \pm 0.22)*10^{-1} rad$ & $(17.9 \pm 5.8)$ & $ (17.20 \pm 0.63) $ \\
\hline $30kHz$ & $(1.13 \pm 0.06) rad$ & $(12.8 \pm 6.9)$ & $ (12.21 \pm 0.74) $ \\
\hline $100kHz$ & $(1.89 \pm 0.10) rad$ & $(3.32 \pm 7.3)$ & $ (2.92 \pm 0.69) $ \\
\hline $300kHz$ & $(3.77 \pm 0.17) rad$ & $(-6.12 \pm 7.38)$ & $ (-8.87 \pm 0.79) $ \\
\hline $1MHz$ & $(3.14 \pm 0.19) rad$ & $(-16.57 \pm 7.38)$ & $ (-26.9 \pm 1.0) $ \\
\hline
\end{tabular}
\end{table}

\subsubsection{Note}
Abbiamo riscontrato un elevato livello di rumore con le misure prese ad una frequenza di 1MHz, già a 300KHz si riscontravano alcuni disturbi.
\subsection{Confronto valori teorici e valori ottenuti dalle misure}
...

\section{Amplificatore invertente}
\subsection{Calcoli teorici}
...
\subsection{Predisposizione scheda}
Per effettuare queste nuove misure è necessario ricofigurare la posizione degli interruttori sulla scheda secondo la seguente configurazione:
\begin{itemize}
	\item S1: Posizione 2
	\item S2: Posizione 2
	\item S3: Posizione 2
	\item S4: Posizione 2
	\item S5: Posizione 2
	\item S6: Posizione 2
	\item S7: Posizione 2
	\item S8: Posizione 2
	\item S9: Posizione 2
\end{itemize}
\subsection{Misura guadagno}
\subsubsection{Impostazione segnale generatore segnale}
Per effettuare la misura di guadagno è richiesto di impostare il generatore di segnale in modo che generi un \textbf{segnale} di uscita \textbf{sinusoidale} di \textbf{frequenza} pari a \textbf{1KHz} e \textbf{valore di picco-picco} di \textbf{1V}.

\subsubsection{Segnale misurato}
Passiamo ora all'analisi dei valori misurati attraverso l'oscilloscopio.\\Dall'analisi dell'uscita visualizzata sul display dell'oscilloscopio possiamo calcolare una \textbf{tensione di ingresso} pari a \large $V_s = (980 \pm 39)$mV \normalsize (in quanto possiamo leggere il risultato come 4.9 divisioni con una scala di 200mV a divisione).\\
Per quanto riguarda la \textbf{tensione di uscita} invece leggiamo un valore di \large $V_u = (10.0 \pm 0.39)$V \normalsize (ottenuto da 5 divisione con una scala di 2.00V per divisione).
\subsubsection{Calcolo guadagno}
Ottemiamo quindi un \textbf{guadagno} pari a:\\ \large $A_v = \frac{V_s}{V_u} = (10.20 \pm 0.80)$ \normalsize \\Che, espresso \textbf{in decibel} diventa:\\ \large $ |A_v|_{dB} = (20.2 \pm 1.6)$dB \normalsize
\subsection{Misura resistenza di ingresso}
Per la misura della resistenza di ingresso procediamo come per l'amplificatore non invertente.
\subsubsection{Misure ottenute}
I valori numerici misurati sono i seguenti:
\begin{itemize}
	\item \large $R_{9} = 10k\Omega \pm 1\%$
	\item \large $V_{u} = (10.0 \pm 0.39)V$
	\item \large $V_{u}^{'} = (6.00 \pm 0.31)V$
\end{itemize}
Inserendoli nella formula precedente otteniamo quindi un valore di: \\ \large $R_{in} = (15.0 \pm 3.5)k\Omega$
\subsection{Misura resistenza di uscita}
Anche per la misura della resistenza di uscita procediamo come per l'amplificatore non invertente.
\subsubsection{Misure ottenute}
I valori numerici misurati sono i seguenti:
\begin{itemize}
	\item \large $R_{10} = 1k\Omega \pm 5\%$
	\item \large $V_{u} = (10.0 \pm 0.39)V$
	\item \large $V_{u}^{'} = (10.0 \pm 0.39)V$
\end{itemize}
\subsubsection{Note}
Si denota vedendo le misure che l'inserimento della resistenza $R_{10}$ è completamente ininfluente per quanto riguarda la tensione di uscita.\\Questo implica che il valore di $R_{10}$ sia completamente trascurabile rispetto a $R_{u}$.
\subsection{Confronto valori teorici e valori ottenuti dalle misure}

\section{Conclusioni}
...

\end{document}
